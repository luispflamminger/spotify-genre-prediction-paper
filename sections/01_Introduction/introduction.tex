\section{Introduction}

Spotify provides the world's leading streaming service.
Since its launch in 2006, the company has gained over 365 million users, of
which nearly 45 percent are subscribed to the chargeable premium service.
The ability to listen to almost any song a user might want
is a great benefit streaming services have over regular music vendors
like iTunes. On the other hand, users might quickly get lost or feel overwhelmed by such
a large collection to choose from.
To guide users and help them find the music they want to listen to in a certain situation,
Spotify uses a number of methods like premade playlists or categories for specific moods and
genres. The biggest benefit to the user experience might certainly be the personalized
playlists, radios or artist and song recommendations.
To be able to have such a robust recommendation system, Spotify needs to understand each user's
listening behaviour and have methods to predict, which music a user might enjoy depending
on their past usage.
Not just user data is important to gain this knowledge, but Spotify also needs to understand
how to categorize music itself and find ways to identify which songs are alike and how they
relate to each other.
Data Science is essential to achieve this goal. Machine learning is used 
to analyze the music in their catalogue and create characteristics about it.
Spotify offers this data to the public, which forms the foundation of the dataset.

\subsection{Problem Definition and Goal}

The paper is set in the context of Big Data Analytics and aims to explain and implement
many of its concepts and processes.
The basic goal of the project is to develop a machine learning model.
During this process, general data mining stages, as layed out in the \ac{CRISP DM},
must be theoretically explored, understood and practically implemented. This includes steps like data
preparation, model creation and evaluation. Creating a model with a high accuracy is only a secondary
concern in this project. Instead gaining a solid understanding of data mining concepts is paramount.

To achieve this goal Spotify song data is collected and an attempt is made to build a Gradient Boosting
Algorithm, which is able to classify songs into different genres to a reasonable accuracy based on preprocessed
audio features from Spotify.

\subsection{Structure and Methodology}

The paper is divided into four main sections, starting with the introduction which contains the problem statement, goal and overall structure.

The second section discusses fundamental concepts. This includes embedding this project into the larger context of Big Data,
explaining the algorithms used on a theoretical level and introducing further concepts which are used for data preparation and modeling.
Additionally, the \ac{CRISP DM} model is explained, which is the basic structure that implementation is based on.
The second section is concluded by an introduction to the use cases and funcionality of Web \acp{API} and a discussion of basic concepts
in music theory.

The third section begins with the data collection process and continues on with the practical implementation steps 
of understanding the dataset's features and labels in the context of Spotify's analysis, preparing and analyzing the data for modeling, creating a
gradient boosting classifier and finally evaluating the project results.

The paper concludes with a summary of the insights gained and embedding the project into a broader context.

Facts presented as part of the paper's fundamental sections are derived from literature research using renowned book sources and
scientific papers. Additionally online articles were used to round out the research.
Most visualizations in this paper are made by hand using Python libraries such as \emph{seaborn} and \emph{matplotlib}.
For data collection and understanding Spotify's developer resources are used as a reference.
Documentation from Python libraries such as scikit learn is used extensively during the preparation, modeling and evaluation parts
of the project.

%Large amounts of data are collected from everyday activities on the platform, stored and
%finally analyzed using machine learning algorithms, to power the recommendation and
%classification process. 


%This Premium
%service comes at a subscription cost of nowadays 12,99C per month, giving the opportu-
%nity to listen to every song that is available at Spotify - this being ca. 70 million tracks from
%over 1,2 million Artists. [54]




%  Based on an existing framework for data science projects. Starts with selection of dataset ...
%- Task: Model development and evaluation using Spotify audio features to classify genres 
%- Simplification: Genre classification instead of specific user recommendations
%- Dataset could be freely chosen and Spotify data was selected
%- Goal is to gain a good understanding about data mining processes, the problems involved
%and possible solutions. The final accuracy of the model is only secondary. 
%
%
%
%
%
%The amount of data collected every day is growing uncompromisingly. Further Proliferation of the Internet, Social Networks, 
%Search queries and increasing networking of IoT devices and sensors means that the amount of data is growing exponentially. 
%To remain competitive in their respective markets, companies must be able to extract value from the large amount of data.
%A conzept that decisively supports companies in exactly this task is Big Data. Big Data encompasses the entire process of data collection to analyzing it
%When combined with machine learning, Big Data unfolds its full potential. The combination allows very large data sets to be processed, 
%which is incredibly valuable for companies. This enables companies to capture their macro and microenvironment in data and create deciding business value. 
%Due to its wide range of applications, this field offers opportunities for improvement for companies of all kinds. 
%Even companies from the entertainment industry, such as Spotify, Netflix or Disney are already using these practices to improve the services they offer, 
%to differentiate themselves from the competition and to make the customer experience unique. 
%Spotify for example uses Big Data to give its users an individual Discover Weekly Playlist which consists of thirty recommended songs for that user. 
%This project has the purpose to illustrate and solve a real-world big data-related problem that could also occur in a company's Value creation process. 
%For this purpose, algorithms are applied to the solution within a controlled framework. Based on the \ac{CRISP DM} process, the data used was first collected, analyzed, 
%and finally evaluated. The problem to be solved is classifying songs using pre-generated features provided by Spotify for each music track on their platform. 
%For the classification, predefined genres by Spotify are taken as categories.
%
%The final goal of this paper is to apply a gradient boosting algorithm to a dataset collected from Spotify. 
%This algorithm should be trained during the course of the project and be able to assign songs to selected genres in the final product. 
%Along the way, the basics of Big Data itself, but also the working models and processes used in the project/model should be presented 
%in their theoretical form and explained clearly. Since not explicitly given, the individual collection of data by means of a given API (written out) 
%in combination with own coding shall find place as part of the model in an extra step. 
%Additionally, this paper should cover all the steps of the \ac{CRISP DM} model, and furthermore be an example of how Big Data projects can be approached 
%and performed in this style of work. Besides the Coding, the mechanisms of the Spotify algorithm and the mechanisms inside Spotify as a company are 
%explicitly explained to guarantee an overall good understanding of the whole project.
