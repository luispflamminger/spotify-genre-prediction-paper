\subsection{Cross Industry Standard Process for Data Mining}

The CRISP-DM Reference Model is an organizational model for data mining projects and provides
an overview of the life cycle of such projects.  
The life cycle of a data mining project is divided into six phases: Business Understanding,
Data Understanding, Data Preparation, Modeling, Evaluation and Deployment. 
The order of the phases is not strictly adhered to. In certain projects,
the outcome of each phase determines which phase or which specific task in a phase must be performed next. \cite[p. 528]{Schroer2021}
However, the process of data mining is to be understood more as a part of the process.
This means that data mining is not necessarily complete once a solution is implemented. 
The lessons that emerge during the process and from the solution can raise new,
often more focused business questions. 
Subsequent data mining processes can benefit from, and often build on,
the lessons learned from previous processes. \cite[p. 529]{Schroer2021}

Firstly, in the step of Business understanding, an in-depth analysis of the business -objectives and -needs has to be done. 
From these insights, the current situation must be accessed and the goals of carrying out the processes must be defined. \cite[]{Lopez2021}
Following with Data Understanding, data is now collected in a targeted or untargeted manner. 
The collected data can be table entries, answers from questionnaires, sequence recordings from machines and so on. 
As a result, the relevant data must be pulled together and made accessible. Additionally, data understanding contains checking the quality of the data.
With unsufficient, or bad data, the model may produce bad or even wrong outcomes. \cite[]{Miner2017}
After the sources are completely identified, proper selection, cleansing, constructing and formatting is done in the step of Data Preparation. 
The data cleaning and quality assurance of the data within the data exploration is the most costly phase of the project. 
It will usually also require the largest share of the available project time, but it is crucial. Here, possible anomalies in the data structure as well as data errors can be identified in advance. 
This phase has a decisive influence on the quality of the overall result. \cite[]{Miner2017}

When it comes to Modeling, the appropriate analytical procedure is chosen to generate predictions and groupings. Connections and structures in the data are also visualized. 
It is important to present these in such a way that the most meaningful visualizations possible are created, which allow the company to make decisions based on evidence-based data.
During the Step of Evaluation, the results of models are evaluated in the backdrop of business intentions. 
Then new objectives may sprout up owing to the new patterns discovered. This is, in fact, an iterative process and the decision whether to consider them or not has 
to be made in this step before moving on to the final phase. \cite[p. 530]{Schroer2021}
The creation of the model is generally not the end of the project. Typically, the knowledge gained must be organized and presented in a way that the customer can use. 
Depending on the requirements, the deployment phase can be as simple as creating a report or as complex as implementing a repeatable data mining process. \cite[p. 532]{Schroer2021}