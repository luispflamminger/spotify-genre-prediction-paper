\subsection{Cross Industry Standard Process for Data Mining}

\ac{CRISP DM} is an organizational model, which provides
an overview of the life cycle of data mining projects.
This cycle is divided into six phases: business understanding,
data understanding, data preparation, modeling, evaluation and deployment. 
The order of the phases is not strictly adhered to. In certain projects,
the outcome of each phase determines which phase or which specific task in a phase must be performed next \cite[p. 528]{Schroer2021}.
However, data mining is not necessarily complete once a solution is implemented. 
After deployment, new insights are gained from the models results and additional data collection,
which can be used to further improve the implementation \cite[p. 529]{Schroer2021}.

The process starts with business understanding, which is an in-depth analysis of business objectives and requirements.
These insights allow an assessment of the current situation and the definition of goals \cite[]{Lopez2021}.
With business understanding complete, data is collected during the data understanding process.
The collected data could come in the form of existing database entries, answers from questionnaires or machine log entries, to name a few examples.
The gathered data is aggregated and examined for usability.
Insufficient or bad data can cause a model to produce unsatisfying or even misleading results \cite[]{Miner2017}.
In the data preparation step, the data is selected, cleaned, formatted and preprocessed to ensure
a high quality dataset for modeling. This is usually the most costly and time intensive phase of the project,
but is crucial for its overall success \cite[]{Miner2017}.

In the next stage, an appropriate model must be developed to generate a result,
which satisfies the requirements defined during business understanding.
During modeling, it is important to constantly present interim results to the management using meaningful visualizations
and subsequently improve the model using further input.

During the step of evaluation, the results of the final model are evaluated in the context of business intentions. 
This might lead to new insights for the organization and in turn alter the goal.
This is an iterative process ending in a final decision to deploy a certain model \cite[p. 530]{Schroer2021}.
Depending on the requirements, the deployment phase could involve creating simple reports,
altering systems or complex data visualizations with a constant reiteration of the data mining process \cite[p. 532]{Schroer2021}.