\subsection{Basic Concepts of Music Theory}

Since this paper aims to use Music streamed on Spotify to create a machine learning model
it is also necessary to briefly discuss the Basics of Music.
Therefore, first a definition for music is given and after that the basic characteristics
of music are analyzed. Secondly methods are shown that help to classify music.

Music itself is a form of art that combines either vocal or instrumental sounds,
sometimes both to express an emotion or convey an idea.\cite{Becker2021WhatIsMusic}
Music plays a key factor in many cultures of the world and is unique and different in each of them.
But despite being able to define music it can be quite difficult even for artists itself to describe
it and they struggle to put their perception into words. Famous Tenor and Saxophonist of
the 1950s/1960s John Coltrane described Music as follows.
"My music is the spiritual expression of what I am my faith, my knowledge, my being.
When you begin to see the possibilities of music, your desire to do something really good for people,
to help humanity free itself from its hang ups. I want to speak to their souls."\cite{Havers2021Sax}

\subsubsection{Melodical Components of Music}
Based on John Coltrane's quote, music could colloquially be described as something,
“that is used to trigger human sense”.\cite{Havers2021Sax}
To achieve this music consists of various elements that work together harmoniously to form
a pleasant-sounding melody.
In music theory there are different opinions about the number of elements,
which again only illustrates how complicated grasping is from the point of view of analysis.
Many factors are highly interdependent and correlate with each other. 
\begin{itemize}
    \item \textbf{Pitch and Melody}: Pitch is the word used to describe the highness or lowness of
    a musical sound.
    A series of Pitches, also described as scales, is used to create a Melody.
    Melodies can be derived from various scales for e.g., the traditional major and minor scales
    of music or even more unusual ones like the whole tone scale.\cite[86]{Hemming2015}
    Furthermore melodies can be described in two ways.
    Conjunct melodies are smooth and easy to sing or play.
    Disjunct melodies however represent the opposite.
    They are ragged or jumpy and difficult to sing or play.

    \item \textbf{Harmony and Chords}: Harmony can be described as the sound created when two or more
    Pitches are performed at the same time to form a chord.
    Chord is simply the definition for three or more notes sounding at once.
    Chords themselves are used to create a musical mood.
    Harmony can be split up into two categories taking the sound of pitches into consideration.
    Harmony can sound Consonant, which means the pitches sound pleasant together or Dissonant,
    meaning the pitches sound unpleasant together.\cite[94]{Hemming2015}

    \item \textbf{Rhythm}: Rhythm refers to the recurrence of notes and silences in time.
    A rhythmic pattern is formed by a series of notes and silence repeats.
    In addition to signify when notes are played, rhythm also defines how long notes are played and
    with what intensity.
    The difference in time creates varying note durations as well as different type of accents.\cite{MilneMusicFundamentals}

    \item \textbf{Texture}: Texture indicates the number of instruments or voices that are used
    to contribute to the density of the music. Texture can be split up into 3 types of Monophonies,
    Homophony and Polyphony.
    Monophony describes a single layer of sound e.g. a solo voice.
    Homophony is a melody with an accompaniment which can be a lead singer with a band or a solo
    singer an a guitar or piano.
    Polyphony is a form of texture which consists of two or more independent voices.
    One voice forms the melody and the other forms a support role.
    This could be for example a lead singer with a choir.\cite{2019ShawMusic}

    \item \textbf{Timbre}: Timbre, also known as tone colour or tone quality,
    can be described as the specific tone or quality an instrument or a voice has.
    Timbre helps to distinguish instruments from each other when playing the same melody or simple notes.
    For example, a “C” Note on the Piano and a sung “C” have the same pitch but different sound
    quality which gets differentiated by timbre.
    Timbre usually can be described with adjectives used to describe color, temperature,
    or the human voice e.g., warm, cold, metallic, harsh, dry.\cite[94]{Hemming2015}

\end{itemize}

\subsubsection{Lyrics and Instruments}

Another way to characterize music is Text. Lyrics are the definition for the linguistic part of a song.
Lyrics usually consist of verses and choruses and can be implicit or explicit.
More or less every song uses unique lyrics but nevertheless we can categorize lyrics by topics they address.\cite{Shipman2014Analysis}
Statistics have shown that the most common themes are Growing Up, Friendship, Statements of Discontent,
heartbreak or Death.\cite{2020AIMMListening}
Another factor that must be taken into consideration when characterizing music are instruments.
In the previous chapter we already talked about the fact that timbre is used to distinguish instruments
from each other.
The timbre focuses mainly on the sound of the instruments to distinguish them.
However, we can distinguish instruments not only by the way they sound but also by how the sound is produced
in the first place.
This is done with a classification of the instruments into 5 categories.

\begin{itemize}
    \item Idiophones: Sound gets produced by the body of the instrument vibrating e.g., xylophones
    \item Membranophones: Sound produced by vibration of a tightly stretched membrane such as drums
    \item Chordophones: Sound produced by vibrating strings e.g., piano or cello
    \item Aerophones: Produce sound by vibrating columns of air such as the pipe organ or oboe
    \item Electrophones: Produce sound by electronic means e.g., synthesizers or electronic instruments.
\end{itemize}

Independently of the tone production, a classification according to the type of player is also possible,
i.e., wind instruments, percussion instruments, string instruments, keyboard instruments,
plucked instruments.\cite{GoshenInstrumentClass}

\subsubsection{Chronological Classification}

Not only compositional aspects can be used to distinguish music,
but also the history of music provides a basis for this.
This is done with the help of epochs.
In music, an epoch is defined as a period in which stylistic similarities prevailed.
However, epoch terms are a bit problematic because they give the impression that different styles
were abruptly replaced.
This is not the case, because with different styles, which often do not resemble each other at all
or even contradict each other, there was often a smooth transition or simultaneous coexistence.
Moreover, epochs are usually only generic terms for many undercurrents.
The epochs that are considered relevant and for the most part include all styles since the 8th century
are the Middle Ages, Renaissance, Baroque, Classicism, Romanticism, and the modern era.\cite{LexikonMusikepochen}
Although the music classifications of the classical and modern eras occur together in history,
there are several differences in distinguishing music.
While classical eras could often be delineated by, for example, the preferred use of instruments,
the increased emergence and sheer diversity in music since the 20th century has made it almost
impossible to bundle all styles under a single epochal term.
For this reason, genres were developed for the classification of so-called "new music".\cite{MusicflxRichtungen}

\subsubsection{Genres}

Musical genres assign different pieces of music with common characteristics to a unified tradition,
history, or convention.
Thus, a genre combines songs that sound the same or emit the same feelings,
as well as songs that are based on the same characteristics of lyrics and instruments in songs.
Dividing music into genres has become standard in today's music industry and due to the diverse
number of possibilities in the creation of songs, there are now already over 1800 recognized specific
genres with various subcategories.
These subcategories or also called subgenres have again basic characteristics of the main
genre as a basis but also their own characteristics to clearly distinguish themselves in their own
genre and are often also often called style of the genre.
These subgenres in turn have subcategories so that somewhat of a "genre tree" can be formed.
The sheer number of genres and subgenres reflects the diversity of music.
In this thesis the focus will be on the 3 main genres Hip Hop, Jazz and Rock which will be discussed
in more detail below.\cite{MusicflxRichtungen}

\textbf{Hip Hop}

Although Hip Hop is often widely considered as a synonym for rap music it can rather be defined as somewhat
of a cultural movement or a form of lifestyle that includes a music genre.
Hip Hop culture consists of various elements which are united under this umbrella term.
Elements are djing/turntablism, rapping, beatboxing, breakdancing, and graffiti/visual art.\cite{MusicalDictHipHop}
These typical street style elements or activities created a cultural revolution and shaped music styles,
fashion, technology, art, and more.\cite{TateHipHop}
Even to this day hip hop continues to develop new art forms that impact the lives of new and old generations.
When it comes to the history of Hip Hop the culture has its origins in the 1970s where it first appeared
in African American ghettos.
Because of this heritage, hip hop still sees itself as street culture, and musicians who practice
this form of music often have close ties to gangs, clans, and poverty.\cite{Rory2019}
Basic elements of hip hop come from the genres of funk and soul, which are also influenced by
African Americans.
Especially because of the distinctive rhythms, these genres offered themselves very well as a basis.
Through the development of hip hop, however, it is also becoming more and more unique in type and form.
The main characteristic of the music of hip hop is the interaction between a rapper and a beat.
Rarely are real instruments used for the music and beats are usually created by electronic
instruments or samples which is, the use of already finished or existing sound/music recordings.
Beats can be very diverse and can represent many moods, e.g., aggressive, relaxed, harsh, etc.\cite{MusicalDictHipHop}
But the main focus in rap is usually on the rapper himself and even more on the lyrics.
Rappers use rhythm, lyrics, and the timbre of their voice to express themselves.
The Artists use their voice as an instrument and more importantly different voice pitches
to adapt to the intention of the lyrics.
Rappers are often measured by the scene on their so-called "flow",
the ability how quickly and how smoothly they can perform chants without errors.
The themes of Hip Hop songs are often poverty, drugs, violence, struggles, or righteousness.
Since many hip-hoppers come from ethnic and racial minorities, they often use personal life
experiences for their lyrics.
In addition, many songs also want to convey a message that often revolves around necessary
political changes, social justice, or grievances in society.\cite{Goodrich2017}

\textbf{Jazz}

Jazz is a musical style in which improvisation plays a central role.
In many jazz performances, artists often play pieces they just made up from their heads and
perform them on spot.
Jazz has its historical roots in the early 19th century America more specifically in New Orleans.
The city was an ideal breeding ground for jazz music because of given cultural diversity.
The percentage of ethnic minorities was much higher in this city than elsewhere in America,
which is why it was often called a melting pot of cultures. African Americans, people from the Caribbean,
European immigrants, and sometimes even white Americans usually lived in the same neighborhoods and racially
segregated ghettos that often existed in other American cities were not present here.\cite{Beek2021Jazz}
However, the biggest influence on Jazz had the Afro-American culture as the music somewhat
reflected the breaking away from previous rules of slavery and oppression.
The improvisation of songs should act symbolically as a counter to the rules they had to deal with for a
long time.
Since its inception, jazz has gone through many different phases.
The beginnings of jazz, from the 1910s to the 1920s, were characterized by small bands,
often consisting of only a frontman and a few accompanists.
The frontman often improvised pieces using a cornet or a trumpet,
while the accompanists supported him with clarinets or trombones.
In addition, often instruments like the banjo, piano, double bass, or drums were used to create a rhythm.
In the 1930s and 1940s, the Swing and Big Band era emerged.
Now, for the first time, singers appeared before big bands and bandleaders,
and the clarinet was largely replaced by the saxophone.
Moreover, the jazz epicenter shifted from New Orleans further and further to New York.\cite{Wildridge2020}
The 1950s and 1960s then introduced laid-back cool jazz in contrast to the more fast-paced songs
of the previous decades.
Here a jazz quartet often played soothing and slow songs. With the introduction of electronic
instruments in jazz from 1970 onwards, many subgenres were formed, such as jazz-rock.
Until today, there are many forms of jazz, which are mainly oriented to the New Orleans origins
and the 3 mentioned eras.
Jazz has its main musical roots in the blues, but there are also elements of rock and classical music in it.\cite{JazzAmHistory}
A distinctive rhythm is a key characteristic of jazz music,
these are created mainly by "swinging" eighth notes.
The so-called swinging is created by emphasizing one note of the eighth note pair while the second
note is lighter and "swings" to the next note.
Jazz is also very polyphonic, which means that many sounds are played simultaneously and as a result,
various layers of harmony are laid over an initial basic melody.
In Addition to the use of classical European instruments such as the saxophone or trumpet,
instruments such as drums, bass, keyboard, guitar, and trombone are often used.
Some types of jazz also have front singers, but often pieces are played without vocals.\cite{2020MasterclassJazz}
As already mentioned before. As already mentioned before the main characteristic of Jazz is
the spirit of improvisation. This unifies almost all forms of jazz music.
All members of a jazz band can be asked to improvise on a jazz tune when performing it.
In addition, jazz artists attach great importance to imprinting their own sound and style on the music,
so they usually even play their own songs slightly modified or with a distinct style.
This leads to the fact that thousands of jazz recordings can be found for the same song,
but they all sound different.
Finally, it is important to mention that jazz can also reflect many different emotions.
Everything from pain to joy is possible. For many People of Colour, it resembles the feeling of freedom,
because for them, as mentioned above, jazz represents a strong voice against suffering, oppression,
and injustice.\cite{MusicalDictJazz}

\textbf{Rock}

The musical Genre Rock is a popular music genre that combines elements of rhythm and blues,
jazz and country music while adding electric instruments.
It originated as Rock `n Roll in the late 1940's and early 1950's and of course also is constantly
changing and evolving.
The basis of rock formed the music genres blues, gospel and country.
Early rock 'n' roll came from cities like Memphis, Chicago, New Orleans or St. Louis.
However, the genre spread very quickly throughout Western culture, and so it was mainly
the British who liked the genre very much and who, in turn, took it to new heights.\cite{2021MasterclassRock}
British bands like the Beatles or the Rolling Stones emerged and became very popular in the USA as well.
Rock dominated the music with its loud, dynamic, energetic and intense style for almost over
50 years until it was replaced by hip hop as the most mainstream genre.
One of the main characteristics of rock is the infectious beat and rhythm.
The music was primarily designed for dancing and was a clear distinction from other music genres
popular at the time, such as jazz or swing.
Thats why from its start in the 50s the genre was very popular among young people as they felt
they could break out of rigid traditions.
Furthermore it is also important to mention that rock, like hip-hop later on,
not only had an impact on music but also a cultural influence on the generations
from the 1950s to the 1990s.\cite{2021MasterclassRock} Rock especially supported the sense of rebellion and social
justice in the western world of the 1960s and influenced clothing style,
hair style and even attitude of its listeners for over 40 years.
The older, more conservative generation at the time, which favored more quiet songs,
rather detested rock.
The energy already mentioned above is a unique characteristic that sets rock apart
from many other genres. Rock music is often very wild, impulsive and driving.
But what defines rock the most is the use of electric instruments and especially
the use of the electric guitar.\cite{MusicalDictRock}
The indispensable electric guitar makes rock probably the genre that is most
influenced by a single instrument. Pioneers of rock like Elvis Presley,
Jimi Hendrix or Chuck Berry experimented a lot with the electric guitar
and used the unique sound to their advantage.
Hendrix in particular often used the so-called "scream" of the electric guitar as his trademark,
where he plucked the string in such a way that a shrill sound could be heard through the amplifier.
This instrument allowed the artists to reach new melodic aspects and pitches that are not achievable
with the use of pure acoustic instruments.\cite{MusicalDictRock} Rock music is typically performed only in bands with a lead singer.
He usually plays an electric guitar himself and is accompanied either by other electric guitars,
normal guitars, other electric instruments and a drum kit.
Lyrical texts of rock are also very diverse due to its division into many subgenres.
Lyrics may not be very profound other lyrics by artists like Bob Dylan, however,
are considered comparable to fine poetry.
Rock music subgenres are very different and vary greatly in terms of rhythm and tempo.
For example, the music genre of heavy metal is almost incomparable with soft rock.\cite{Clark2021}