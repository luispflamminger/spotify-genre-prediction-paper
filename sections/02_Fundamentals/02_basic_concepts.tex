\subsection{Basic Concepts of Big Data}

Big Data is an umbrella term used to describe various technological but also
organizational developments.
Originally, Big Data refers to large sets of structured and unstructured data
which must be stored and processed to gain business value.
Today, Big Data is also often used as buzzword to outline various modern use cases
that deal with large amounts of data. Big Data is therefore often used in conjunction
with other buzzwords like automatization, personalization or monitoring.
This chapter presents the foundation of Big Data in its technical implementation and
combines the topics with busines cases. 

\subsubsection{Relevance of Data}

Data in combination with Business Intelligence become increasingly important
over the past decades and is closely associated with the advances of the internet
itself.
Looking back, Business Intelligence can be divided into three sub-categories,
which follow another linearly. The first phase is centered around getting critical
insights into operations from structured data gathered while running the business
and interacting with customers. Examples would be transactions and sales.
The second phase focuses increasingly on data mining and gathering customer-specific
data. These insights can be used to identify customer needs, opinions and interests.
The third phase, often referred as Big Data, enhances the focus set in phase
two by more features and much deeper analysis possibilities.
It allows to gain critical information such as location,person,
context often through mobile and sensor-based context. 

In conclusion, organizations require Business Intelligence as it allows them to gain
crucial insights which is needed to run the business and achieve an advantage
over the competition.
It is important to minimize the uncertainty of decisions
and maximize the knowledge about the opportunity costs and derive their intended impacts. 
It is clearly noticeable that the insights and analysis possibilities become
progressively deeper and much more detailed.
Along this trend the amount of data required becomes larger and larger with
increasingly complex data structures. Size, complexity of data and deep analysis
form the foundation of Big Data and can be found again in the 5V matrix of Big Data. 

\subsubsection{The 5V Matrix for Big Data}
When describing Data, a reference is often made to the five Vs,
which highlight its main characteristics.
The previous aspects of Big Data can again be recognized in averted form. 

\textbf{Volume:} The size of the datasets is in the range of tera- and zettabyte
. This massive volume is not a challenge for storing but also extracting
relevant information from the mass of data. 

\subsubsection{Reinforcement Learning}

\subsubsection{Machine Learning Algorithms}