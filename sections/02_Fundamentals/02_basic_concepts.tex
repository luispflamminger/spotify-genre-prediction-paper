\subsection{Basic Concepts of Big Data}

Big Data is an umbrella term used to describe various technological but also
organizational developments.
Originally, Big Data refers to large sets of structured and unstructured data
which must be stored and processed to gain business value.
Today, Big Data is also often used as buzzword to outline various modern use cases
that deal with large amounts of data. Big Data is therefore often used in conjunction
with other buzzwords like automatization, personalization or monitoring.
This chapter presents the foundation of Big Data in its technical implementation and
combines the topics with busines cases. 

\subsubsection{Relevance of Data}

Data in combination with Business Intelligence become increasingly important
over the past decades and is closely associated with the advances of the internet
itself.
Looking back, Business Intelligence can be divided into three sub-categories,
which follow another linearly. The first phase is centered around getting critical
insights into operations from structured data gathered while running the business
and interacting with customers. Examples would be transactions and sales.
The second phase focuses increasingly on data mining and gathering customer-specific
data. These insights can be used to identify customer needs, opinions and interests.
The third phase, often referred as Big Data, enhances the focus set in phase
two by more features and much deeper analysis possibilities.
It allows to gain critical information such as location,person,
context often through mobile and sensor-based context. 

In conclusion, organizations require Business Intelligence as it allows them to gain
crucial insights which is needed to run the business and achieve an advantage
over the competition.
It is important to minimize the uncertainty of decisions
and maximize the knowledge about the opportunity costs and derive their intended impacts. 
It is clearly noticeable that the insights and analysis possibilities become
progressively deeper and much more detailed.
Along this trend the amount of data required becomes larger and larger with
increasingly complex data structures. Size, complexity of data and deep analysis
form the foundation of Big Data and can be found again in the 5V matrix of Big Data. 

\subsubsection{The 5V Matrix for Big Data}
When describing Data, a reference is often made to the five Vs,
which highlight its main characteristics.
The previous aspects of Big Data can again be recognized in averted form. 

\textbf{Volume:} The size of the datasets is in the range of tera- and zettabyte
. This massive volume is not a challenge for storing but also extracting
relevant information from the mass of data. 

\subsubsection{Reinforcement Learning}

\subsubsection{Categories of Machine Learning}

Machine learning refers to the ability of a computer to learn on its own.
The field of machine learning includes a wide range of algorithms that learn
from data and make predictions with variable quality.
These predictions are not made programmatically but by data-driven predictions
that are "learned" by generating knowledge. Basically,
Machine Learning can be divided into 3 different methods which are discussed below.\cite[4]{2018VDMAQuick}

\textbf{Supervised Learning}
In supervised learning, a function is determined by mapping input values to known target
values using examples. Supervised learning is also called learning from examples for this reason.\cite[96]{schacht2019blockchain}
The system learns based on a training data set that already contains the correct answers.
Using the already given data sets, the algorithm learns to set up rules and patterns to
reach the known target variable. The process is repeated until the prediction matches the desired quality.
Experiences from each iteration are in turn included in the learning process.
If the trained model fulfills the desired results, it can also be applied to unknown data.
The goal of this learning method is therefore to make predictions and recommendations.\cite[96]{schacht2019blockchain}
It is also important to mention that target values in machine learning are called labels.
Here again, one must distinguish between two use cases. If the label is nominal, it is a so-called
classification. So, the model should assign data to specific classes.
A regression is present if the label has metric target values.\cite[46]{Paass2020}
The aim regression is to use data to make forecasts of future values or to identify trends.
Problems that can occur when using supervised learning and should be avoided if possible
are either overfitting or underfitting.  Overfitting is when the algorithm is adapted too much
to the training data.
This means that it delivers better results when applied to training data than to unknown data.
The opposite is called underfitting which means the models of the learning procedure are not complex
enough.
Therefore, the algorithm cannot deliver sufficient performance to make predictions.\cite{buxmann2018künstliche}

\textbf{Unsupervised Learning}
The opposite of supervised learning is unsupervised learning.
Here, the algorithm itself acquires patterns and correlations without explicitly
predefining target values.
Which is why the algorithm is also called learning from observations.\cite[97]{schacht2019blockchain}
This is also done using sample data, but without labeled output data as in supervised learning.
The algorithm thus focuses on deriving or recognizing common structures and patterns in the sample data.\cite[7]{2018VDMAQuick}
The search includes the complete training data and not only relations with concrete
target values as in supervised learning.\cite[802]{ernst2016grundkurs}
Unsupervised learning can again be divided into several types.
Clustering, for example, deals with finding frequency structures, patterns and grouping the
data into a few sets.\cite[260]{Ertel2021}
Another type of learning are associations which search for rules that map connections between data points.
Finally, there is also dimensionality reduction. This kind of learning tries to reduce
all available variables to the most important ones.\cite[10]{FraunhoferMasch2018}
The problem with unsupervised learning, however, is that no conclusions can be drawn about
the quality of the algorithm due to the non-existence of target variables.
Thus, there is no real right or wrong in the learning process and so the algorithm has the possibility
to fail completely.\cite[97]{schacht2019blockchain}

\textbf{Semi Supervised Learning}
Semi-supervised learning is a learning method that includes elements of supervised
and unsupervised learning.
Because of this combination, it is often not considered a separate form of learning,
but it is important to mention it. This method uses training data that are only partially labeled.\cite[98]{schacht2019blockchain}
This training data serves as a representative label of discovered structures if unsupervised learning is used.
If supervised methods are used, the unlabeled data serves to make statistical accumulations more estimable.
The main advantage of semi-supervised learning is that
training of algorithms is already possible with little data.
This reduces the high costs and the effort that often have to be spent on target values.\cite{WuttkeDatasolutMachine}

\textbf{Reinforcement Learning}
Reinforcement learning is fundamentally different from the learning methods already mentioned.
Unlike the learning methods already discussed, no training data is available at the beginning.\cite[351]{Ertel2021}
The algorithm acquires data only by interacting with the environment,
which is why this learning method can also be called learning by interaction.
Due to the few Requirements, this learning method is optimally suited\cite[98]{schacht2019blockchain}
Models train by tackling challenging problems and having their decisions immediately rewarded
if successful or punished if unsuccessful through feedback signals.
This reward or punishment occurs, for example, in chess when a game is won or lost.
Moves that lead to victory are saved, as are moves that lead to a possible loss.\cite[98]{schacht2019blockchain}
In contrast to other learning methods, the algorithm does not know whether a decision is right or
wrong before it decides what to do next.
In addition, it does not know whether the decision it is currently making is the best one for the situation at hand,
since its wealth of experience only grows with increasing runtime. In order to cover all possible situations,
the algorithm must initially also cover previously unknown possibilities and not only act on the basis of actions
that have led to reward or punishment in the past.
However, if the algorithm is trained well enough, it can solve problems very well on this basis.
Overall goal is to develop a strategy to maximize rewards received and get better as fast as possible.\cite[351]{Ertel2021}