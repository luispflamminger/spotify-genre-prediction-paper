\section{Conclusion}

- Ziel dieser Arbeit 

    - Klassifikation mithilfe von Gradient Boosting 
    - Aus selbstgenerierten Daten -> Spotify

    - Ausprobieren > bestes Modell 

- Modell und Projekt liefern gute Ergebnisse 

    - Anfang mit Decision Trees und danach Gradient Boosting hat sich als guter weg erwiesen
        -> Basis mithilfe derer Gradient Boosting aufgebaut werden konnte 

    - Keine Optimierungsmethoden außer Standardisierung brachten keine Verbesserungen
        - interessant und deutet auf gute Datenbasis hin 
        - Eventuell falsche Implementation und Ansatz, da noch viele Details offen sind (wie bereits besprochen in Evaluation)

    - Modelle beruhen auf ähnlichen Features, die bereits mittels der Theorie ermittelt wurden. 
        - Wie gut Modelle sind und wie sehr theoretisches nachgebildet werden kann
        - lässt sich vermutlich auch auf andere Machine Learning Algorithmen anwenden 
            - Analyse von komplexen, nicht eindeutigen und subjektive Daten möglich mit guten Resultaten 

-   Persönliche Erkenntnisse
    - Trotz einfachem Modell und Daten an Grenzen des Projektrahmens gestoßen -> Anzahl Wörter, Zeit, Modelle, etc. 
        - Planung zu Beginn konnte nicht genau eingehalten werdne 

    - Der Prozess von Datensammlung zu Modell war nicht geradlinig, aber hat durchaus sehr gut funktioniert und zu den 
      gewünschten Resultaten geführt 

    - Implementation nicht so kompliziert ist und viel Freude bereitet
    - Praktische Projekte öffnen augen in vielerlei Hinsicht. 
    - Teamarbeit zwar teilweise anstrengend und komplexer war, aber auch mehr Ideen, Input und allgemein bessere Ergebnisse hervorbrachte 
        - keiner hatte bisher Erfahrung mit Machine Learning -> Unterstützung  + Zusammenarbeit (über eigene Teile hinweg)

        Since the problems and the assessment of the project work have already been described to a large extent in the evaluation, 
        this chapter focusses more on the organizational and personal conclusions we experienced. 

        The project had the purpose to classify a dataset with the help of a machine learning algorithm. With no restrictions on the choice of dataset and algorithm given, 
        we quickly unionized over the idea to implement a classification of genres based on Spotify Music data. After an extended search, 
        no sufficient existing dataset was found and thus a new dataset had to be created. Even though the data collection was complex at times, 
        it proofed to be a good decision and convinces with its novelty. After testing and debating over various algorithms, Gradient Boosting with 
        a pre-classification done via a decision tree as its basis turned out to be the most suitable algorithm for the goal of the project.
        Overall, the model works successful, delivering a confidence of nearly 90 percent. Beginning with Decision Trees to gain insides and a feeling of the data came in very handy. 
        On this basis the gradient boosting algorithm could be implemented more easily and achieve better results. Interestingly, except Standardization, 
        no optimization method led to a better result in the end. This could indicate a high quality of the dataset. 
        
        Weitere Ausführungen

        Personally, several lessons were learned out of this project. On the one hand, we tended to reach the formal and organizational limits. 
        For example, in terms of time, since the algorithm was sometimes worked on too precisely, we lost scope of the main goal, focusing too much on incidentals. 
        In the future, it will be worthwhile to set the framework more precisely and, above all, more appropriately. Although implementation often is a lot of fun, 
        the goal has to be be achieved in time and to the extent intended. Teamwork certainly also plays an important role here. As expected, due to teamwork, 
        problems often occur cause of different approaches. But for us, the thematic cooperation is particularly noteworthy. The topics and chapters were divided up in advance, 
        but there was still a constant exchange of information on different issues in the various chapters. This makes the topics more complex, but also the introduction and explanations more rounded.

        In conclusion, Spotify is just one example of many modern applications that use Big Data to capture and attempt to reflect the often subjective world in numbers. 
        What is particularly interesting here is how this is attempted, and to what extent it works. The success of the company, or the widespread use of the app, 
        show how well the Echo Nest works. Still, like the model shown here, the deciding attributes are the ones closer to numbers, like Accousticness or Speechiness. 
        Attributes such as Valence, which attempts to measure mood, plays a small role in genre assignment. However, this can, and will, change in the future. 
        The decoding of the personal feeling of the user is the next step to more, and above all more decisive data and in the broader sense of a better customer relationship. 
        So we can look forward with great interest to the further development of such use cases.

