\section{Conclusion}
        %Since the problems and the assessment of the project work have already been described to a large extent in the evaluation, 
        %this chapter focusses more on the organizational and personal conclusions we experienced. 

        The project had the purpose to classify a dataset with the help of a machine learning algorithm. With no restrictions on the choice of dataset and algorithm given, 
        a classification of genres based on Spotify Music data was selected as a project task. After an extended search, 
        no sufficient existing dataset was found and thus a new dataset had to be created. Even though the data collection was complex at times, 
        it proofed to be a good decision and convinces with its novelty. After testing and debating over various algorithms, gradient boosting was
        selected as the machine learning algorithm. Additionally, a classification tree was implemented as a reference point.

        Overall the result was successful, with an accuracy of nearly 87 percent, with all other evaluation metrics indicating to a similar positive performance.
        Interestingly, despite high optimization efforts, only standardization led to a better result with no other optimization method improving the quality of the model. 
        This could indicate either that these methods were implemented incorrectly, or that the quality of the dataset was very high from the beginning. 
        The latter is supported by the fact that both algorithms classified the songs according to similar features as stated in theory to distinguish between genres.

        Personally, several lessons were learned from this project. The first lesson is, that despite only focussing on a 
        simple machine learning algorithm, the limits of the project framework were reached. Both in terms of the scope and size of the
        paper itself and from a algorithm implementation perspective, as mulitple optimization possibilites and design choices were 
        discovered while implementing and had to be addressed further. This goes hand in hand with the fact that the project execution was not straight-forward.
        With multiple changes and additions to the paper as well as implementations made along the way which shaped the 
        project into its final form. However, all changes should in no way be seen as negative, as they have led to the desired result and 
        are part of the learning process.
        Secondly, teamwork was required to manage a project of this size. The topics and chapters were assigned in advance, with constant exchange 
        of information on different issues and collaboration on topics such as data collection and model implementation.
        Thirdly, the overall learning curve was very steep. No participant of this group had previous experience in machine learning.
        Nevertheless, it was possible to create a well working classification model which meets the pre-set conditions of the overall task.

        - mittlerweile unglaublich einfach sowas zu selbst ohne großen Aufwand zu erstellen

        In conclusion, Spotify is just one example of many modern applications that use Big Data analytics to capture and attempt to reflect the often subjective world in numbers. 
        What is particularly interesting here is how this is attempted, and to what extent it works. The success of the company, or the widespread use of the app, 
        show how well the Echo Nest works. Still, like the model shown here, the deciding attributes are the ones closer to numbers, like Accousticness or Speechiness. 
        Attributes such as Valence, which attempts to measure mood, plays a small role in genre assignment. However, this can, and will, change in the future. 
        The decoding of the personal feeling of the user is the next step to more, and above all more decisive data and in the broader sense of a better customer relationship. 
        So we can look forward with great interest to the further development of such use cases.

