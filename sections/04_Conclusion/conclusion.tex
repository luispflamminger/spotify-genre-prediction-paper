\section{Conclusion}

- Ziel dieser Arbeit 

    - Klassifikation mithilfe von Gradient Boosting 
    - Aus selbstgenerierten Daten -> Spotify

    - Ausprobieren > bestes Modell 

- Modell und Projekt liefern gute Ergebnisse 

    - Anfang mit Decision Trees und danach Gradient Boosting hat sich als guter weg erwiesen
        -> Basis mithilfe derer Gradient Boosting aufgebaut werden konnte 

    - Keine Optimierungsmethoden außer Standardisierung brachten keine Verbesserungen
        - interessant und deutet auf gute Datenbasis hin 
        - Eventuell falsche Implementation und Ansatz, da noch viele Details offen sind (wie bereits besprochen in Evaluation)

    - Modelle beruhen auf ähnlichen Features, die bereits mittels der Theorie ermittelt wurden. 
        - Wie gut Modelle sind und wie sehr theoretisches nachgebildet werden kann
        - lässt sich vermutlich auch auf andere Machine Learning Algorithmen anwenden 
            - Analyse von komplexen, nicht eindeutigen und subjektive Daten möglich mit guten Resultaten 

-   Persönliche Erkenntnisse
    - Trotz einfachem Modell und Daten an Grenzen des Projektrahmens gestoßen -> Anzahl Wörter, Zeit, Modelle, etc. 
        - Planung zu Beginn konnte nicht genau eingehalten werdne 

    - Der Prozess von Datensammlung zu Modell war nicht geradlinig, aber hat durchaus sehr gut funktioniert und zu den 
      gewünschten Resultaten geführt 

    - Implementation nicht so kompliziert ist und viel Freude bereitet
    - Praktische Projekte öffnen augen in vielerlei Hinsicht. 
    - Teamarbeit zwar teilweise anstrengend und komplexer war, aber auch mehr Ideen, Input und allgemein bessere Ergebnisse hervorbrachte 
        - keiner hatte bisher Erfahrung mit Machine Learning -> Unterstützung  + Zusammenarbeit (über eigene Teile hinweg)

The project had the purpose to classify a dataset with the help of a machine learning algorithm. With no restrictions on the 
choice of dataset and algroithm. The choice made to implement a classifcation of genres based on Spotify Music data. After an 
extended search sufficient existing dataset was found and thus a new dataset had to be created. Even though the data collection 
was complex at times, it proofed to be a good decision and convinces with its novelty. After the completion of the dataset, a 
suitable algorithm had to be found. ... Choice fell on ... Gradient Boosting with a pre-classification done via a decision tree
as it basis. 
